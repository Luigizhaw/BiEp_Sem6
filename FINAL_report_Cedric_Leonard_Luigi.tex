% Options for packages loaded elsewhere
\PassOptionsToPackage{unicode}{hyperref}
\PassOptionsToPackage{hyphens}{url}
%
\documentclass[
]{article}
\usepackage{amsmath,amssymb}
\usepackage{iftex}
\ifPDFTeX
  \usepackage[T1]{fontenc}
  \usepackage[utf8]{inputenc}
  \usepackage{textcomp} % provide euro and other symbols
\else % if luatex or xetex
  \usepackage{unicode-math} % this also loads fontspec
  \defaultfontfeatures{Scale=MatchLowercase}
  \defaultfontfeatures[\rmfamily]{Ligatures=TeX,Scale=1}
\fi
\usepackage{lmodern}
\ifPDFTeX\else
  % xetex/luatex font selection
\fi
% Use upquote if available, for straight quotes in verbatim environments
\IfFileExists{upquote.sty}{\usepackage{upquote}}{}
\IfFileExists{microtype.sty}{% use microtype if available
  \usepackage[]{microtype}
  \UseMicrotypeSet[protrusion]{basicmath} % disable protrusion for tt fonts
}{}
\makeatletter
\@ifundefined{KOMAClassName}{% if non-KOMA class
  \IfFileExists{parskip.sty}{%
    \usepackage{parskip}
  }{% else
    \setlength{\parindent}{0pt}
    \setlength{\parskip}{6pt plus 2pt minus 1pt}}
}{% if KOMA class
  \KOMAoptions{parskip=half}}
\makeatother
\usepackage{xcolor}
\usepackage[margin=1in]{geometry}
\usepackage{color}
\usepackage{fancyvrb}
\newcommand{\VerbBar}{|}
\newcommand{\VERB}{\Verb[commandchars=\\\{\}]}
\DefineVerbatimEnvironment{Highlighting}{Verbatim}{commandchars=\\\{\}}
% Add ',fontsize=\small' for more characters per line
\usepackage{framed}
\definecolor{shadecolor}{RGB}{248,248,248}
\newenvironment{Shaded}{\begin{snugshade}}{\end{snugshade}}
\newcommand{\AlertTok}[1]{\textcolor[rgb]{0.94,0.16,0.16}{#1}}
\newcommand{\AnnotationTok}[1]{\textcolor[rgb]{0.56,0.35,0.01}{\textbf{\textit{#1}}}}
\newcommand{\AttributeTok}[1]{\textcolor[rgb]{0.13,0.29,0.53}{#1}}
\newcommand{\BaseNTok}[1]{\textcolor[rgb]{0.00,0.00,0.81}{#1}}
\newcommand{\BuiltInTok}[1]{#1}
\newcommand{\CharTok}[1]{\textcolor[rgb]{0.31,0.60,0.02}{#1}}
\newcommand{\CommentTok}[1]{\textcolor[rgb]{0.56,0.35,0.01}{\textit{#1}}}
\newcommand{\CommentVarTok}[1]{\textcolor[rgb]{0.56,0.35,0.01}{\textbf{\textit{#1}}}}
\newcommand{\ConstantTok}[1]{\textcolor[rgb]{0.56,0.35,0.01}{#1}}
\newcommand{\ControlFlowTok}[1]{\textcolor[rgb]{0.13,0.29,0.53}{\textbf{#1}}}
\newcommand{\DataTypeTok}[1]{\textcolor[rgb]{0.13,0.29,0.53}{#1}}
\newcommand{\DecValTok}[1]{\textcolor[rgb]{0.00,0.00,0.81}{#1}}
\newcommand{\DocumentationTok}[1]{\textcolor[rgb]{0.56,0.35,0.01}{\textbf{\textit{#1}}}}
\newcommand{\ErrorTok}[1]{\textcolor[rgb]{0.64,0.00,0.00}{\textbf{#1}}}
\newcommand{\ExtensionTok}[1]{#1}
\newcommand{\FloatTok}[1]{\textcolor[rgb]{0.00,0.00,0.81}{#1}}
\newcommand{\FunctionTok}[1]{\textcolor[rgb]{0.13,0.29,0.53}{\textbf{#1}}}
\newcommand{\ImportTok}[1]{#1}
\newcommand{\InformationTok}[1]{\textcolor[rgb]{0.56,0.35,0.01}{\textbf{\textit{#1}}}}
\newcommand{\KeywordTok}[1]{\textcolor[rgb]{0.13,0.29,0.53}{\textbf{#1}}}
\newcommand{\NormalTok}[1]{#1}
\newcommand{\OperatorTok}[1]{\textcolor[rgb]{0.81,0.36,0.00}{\textbf{#1}}}
\newcommand{\OtherTok}[1]{\textcolor[rgb]{0.56,0.35,0.01}{#1}}
\newcommand{\PreprocessorTok}[1]{\textcolor[rgb]{0.56,0.35,0.01}{\textit{#1}}}
\newcommand{\RegionMarkerTok}[1]{#1}
\newcommand{\SpecialCharTok}[1]{\textcolor[rgb]{0.81,0.36,0.00}{\textbf{#1}}}
\newcommand{\SpecialStringTok}[1]{\textcolor[rgb]{0.31,0.60,0.02}{#1}}
\newcommand{\StringTok}[1]{\textcolor[rgb]{0.31,0.60,0.02}{#1}}
\newcommand{\VariableTok}[1]{\textcolor[rgb]{0.00,0.00,0.00}{#1}}
\newcommand{\VerbatimStringTok}[1]{\textcolor[rgb]{0.31,0.60,0.02}{#1}}
\newcommand{\WarningTok}[1]{\textcolor[rgb]{0.56,0.35,0.01}{\textbf{\textit{#1}}}}
\usepackage{longtable,booktabs,array}
\usepackage{calc} % for calculating minipage widths
% Correct order of tables after \paragraph or \subparagraph
\usepackage{etoolbox}
\makeatletter
\patchcmd\longtable{\par}{\if@noskipsec\mbox{}\fi\par}{}{}
\makeatother
% Allow footnotes in longtable head/foot
\IfFileExists{footnotehyper.sty}{\usepackage{footnotehyper}}{\usepackage{footnote}}
\makesavenoteenv{longtable}
\usepackage{graphicx}
\makeatletter
\def\maxwidth{\ifdim\Gin@nat@width>\linewidth\linewidth\else\Gin@nat@width\fi}
\def\maxheight{\ifdim\Gin@nat@height>\textheight\textheight\else\Gin@nat@height\fi}
\makeatother
% Scale images if necessary, so that they will not overflow the page
% margins by default, and it is still possible to overwrite the defaults
% using explicit options in \includegraphics[width, height, ...]{}
\setkeys{Gin}{width=\maxwidth,height=\maxheight,keepaspectratio}
% Set default figure placement to htbp
\makeatletter
\def\fps@figure{htbp}
\makeatother
\setlength{\emergencystretch}{3em} % prevent overfull lines
\providecommand{\tightlist}{%
  \setlength{\itemsep}{0pt}\setlength{\parskip}{0pt}}
\setcounter{secnumdepth}{-\maxdimen} % remove section numbering
\ifLuaTeX
  \usepackage{selnolig}  % disable illegal ligatures
\fi
\usepackage{bookmark}
\IfFileExists{xurl.sty}{\usepackage{xurl}}{} % add URL line breaks if available
\urlstyle{same}
\hypersetup{
  pdftitle={Exploring Socioeconomic Factors and Long COVID in U.S. Adults (18--65)},
  pdfauthor={Leonard, Cedric, Luigi},
  hidelinks,
  pdfcreator={LaTeX via pandoc}}

\title{Exploring Socioeconomic Factors and Long COVID in U.S. Adults
(18--65)}
\author{Leonard, Cedric, Luigi}
\date{2025-05-05}

\begin{document}
\maketitle

\begin{center}\rule{0.5\linewidth}{0.5pt}\end{center}

\subsubsection{Introduction}\label{introduction}

The COVID-19 pandemic has had massive health and societal impacts around
the world. However, its long-term effects --- especially \textbf{Long
COVID} --- are still being studied and better understood.

Long COVID refers to symptoms that persist for weeks or even months
after initial recovery from the virus. These symptoms can include
fatigue, shortness of breath, brain fog, and others that significantly
reduce quality of life. For some people, it can mean missing work,
struggling with daily activities, and increased medical expenses.

In this project, we asked:

\begin{quote}
\textbf{Do socioeconomic factors such as income, education, and access
to healthcare influence the prevalence and severity of Long COVID
symptoms among U.S. adults aged 18--65?}
\end{quote}

\begin{center}\rule{0.5\linewidth}{0.5pt}\end{center}

\subsubsection{Why This Question?}\label{why-this-question}

We chose this topic because Long COVID is more than just a medical
condition --- it's a public health issue. People with fewer resources or
access to care may be more likely to:

\begin{itemize}
\tightlist
\item
  Be exposed to COVID-19 due to working in frontline jobs\\
\item
  Have untreated pre-existing conditions\\
\item
  Delay care because of cost or lack of insurance\\
\item
  Live in environments with limited healthcare access
\end{itemize}

As part of our \textbf{Biostatistics and Epidemiology module}, we wanted
to apply statistical techniques that allow us to answer real-world
public health questions --- not just whether Long COVID is happening,
but \textbf{who is most affected}, and \textbf{how bad} the impact is.

\begin{center}\rule{0.5\linewidth}{0.5pt}\end{center}

\subsubsection{Data Source}\label{data-source}

We used the 2023--2024 \textbf{National Health Interview Survey (NHIS)}
COVID-19 supplements from the CDC. These data are publicly available and
contain thousands of responses from adults across the United States.

The dataset includes:

\begin{itemize}
\tightlist
\item
  Demographic information (e.g., age, sex, education)
\item
  COVID-19 infection and symptom history
\item
  Questions on healthcare access and affordability
\item
  Income, poverty ratio, and household characteristics
\end{itemize}

After cleaning, we focused on respondents between ages \textbf{18 and
65}, since older adults (65+) have different risk profiles and access to
Medicare, which could skew the analysis.

\begin{center}\rule{0.5\linewidth}{0.5pt}\end{center}

\subsubsection{Variables We Used}\label{variables-we-used}

Here's a table showing which variables we kept and what they mean:

\begin{longtable}[]{@{}
  >{\raggedright\arraybackslash}p{(\columnwidth - 2\tabcolsep) * \real{0.1720}}
  >{\raggedright\arraybackslash}p{(\columnwidth - 2\tabcolsep) * \real{0.8280}}@{}}
\toprule\noalign{}
\begin{minipage}[b]{\linewidth}\raggedright
Variable
\end{minipage} & \begin{minipage}[b]{\linewidth}\raggedright
Description
\end{minipage} \\
\midrule\noalign{}
\endhead
\bottomrule\noalign{}
\endlastfoot
\texttt{LONGCOVD1\_A} & Has the respondent ever had Long COVID symptoms?
(1 = Yes, 2 = No) \\
\texttt{LCVDACT\_A} & Does Long COVID reduce daily activity? (1 = Yes, 2
= No, 3 = Some days) \\
\texttt{POVRATTC\_A} & Poverty ratio: income / poverty threshold (0 =
poorest, 11 = richest) \\
\texttt{RATCAT\_A} & Income category from low (1) to high (14) \\
\texttt{EDUCP\_A} & Education level: 1 = no school, 10 = doctoral
degree \\
\texttt{USUALPL\_A} & Usual place for medical care (1 = Yes, 2 = No, 3 =
More than one) \\
\texttt{MEDDL12M\_A} & Delayed care in last 12 months due to cost? (1 =
Yes, 2 = No) \\
\texttt{MEDNG12M\_A} & Needed care but couldn't get it? (1 = Yes, 2 =
No) \\
\texttt{TRANSPOR\_A} & Couldn't get care due to transportation? (1 =
Yes, 2 = No) \\
\texttt{AGEP\_A} & Age of respondent (filtered to 18--65 years) \\
\texttt{SEX\_A} & Sex of respondent (1 = Male, 2 = Female) \\
\end{longtable}

These variables were chosen to reflect different dimensions of
\textbf{socioeconomic status (SES)} and \textbf{access to healthcare}
--- the main factors in our hypothesis.

\begin{center}\rule{0.5\linewidth}{0.5pt}\end{center}

\subsubsection{Data Loading and
Filtering}\label{data-loading-and-filtering}

We began by loading the raw CSV file, keeping only the relevant
variables, and applying basic filters to clean up invalid values.

\begin{Shaded}
\begin{Highlighting}[]
\CommentTok{\# Load the dataset}
\NormalTok{data }\OtherTok{\textless{}{-}} \FunctionTok{read.csv}\NormalTok{(}\StringTok{"C:/Users/Luigi/OneDrive/ZHAW/6. Semester/BiEp\_Sem6/adult23.csv"}\NormalTok{)}

\CommentTok{\# Keep relevant variables}
\NormalTok{relevant\_vars }\OtherTok{\textless{}{-}} \FunctionTok{c}\NormalTok{(}
  \StringTok{"AGEP\_A"}\NormalTok{, }\StringTok{"LONGCOVD1\_A"}\NormalTok{, }\StringTok{"LCVDACT\_A"}\NormalTok{, }\StringTok{"POVRATTC\_A"}\NormalTok{, }\StringTok{"RATCAT\_A"}\NormalTok{,}
  \StringTok{"EDUCP\_A"}\NormalTok{, }\StringTok{"USUALPL\_A"}\NormalTok{, }\StringTok{"MEDDL12M\_A"}\NormalTok{, }\StringTok{"MEDNG12M\_A"}\NormalTok{, }\StringTok{"TRANSPOR\_A"}\NormalTok{, }\StringTok{"SEX\_A"}
\NormalTok{)}

\NormalTok{filtered\_data }\OtherTok{\textless{}{-}}\NormalTok{ data }\SpecialCharTok{\%\textgreater{}\%}
  \FunctionTok{select}\NormalTok{(}\FunctionTok{all\_of}\NormalTok{(relevant\_vars))}
\end{Highlighting}
\end{Shaded}

Next, we filtered for only valid responses and created new variables
where needed.

\begin{Shaded}
\begin{Highlighting}[]
\NormalTok{filtered\_clean }\OtherTok{\textless{}{-}}\NormalTok{ filtered\_data }\SpecialCharTok{\%\textgreater{}\%}
  \FunctionTok{filter}\NormalTok{(}
\NormalTok{    AGEP\_A }\SpecialCharTok{\textgreater{}=} \DecValTok{18} \SpecialCharTok{\&}\NormalTok{ AGEP\_A }\SpecialCharTok{\textless{}=} \DecValTok{65}\NormalTok{,}
\NormalTok{    LONGCOVD1\_A }\SpecialCharTok{\%in\%} \FunctionTok{c}\NormalTok{(}\DecValTok{1}\NormalTok{, }\DecValTok{2}\NormalTok{),}
\NormalTok{    LCVDACT\_A }\SpecialCharTok{\%in\%} \FunctionTok{c}\NormalTok{(}\DecValTok{1}\NormalTok{, }\DecValTok{2}\NormalTok{, }\DecValTok{3}\NormalTok{),}
\NormalTok{    EDUCP\_A }\SpecialCharTok{\%in\%} \DecValTok{1}\SpecialCharTok{:}\DecValTok{10}\NormalTok{,}
\NormalTok{    USUALPL\_A }\SpecialCharTok{\%in\%} \FunctionTok{c}\NormalTok{(}\DecValTok{1}\NormalTok{, }\DecValTok{2}\NormalTok{, }\DecValTok{3}\NormalTok{),}
\NormalTok{    MEDDL12M\_A }\SpecialCharTok{\%in\%} \FunctionTok{c}\NormalTok{(}\DecValTok{1}\NormalTok{, }\DecValTok{2}\NormalTok{),}
\NormalTok{    MEDNG12M\_A }\SpecialCharTok{\%in\%} \FunctionTok{c}\NormalTok{(}\DecValTok{1}\NormalTok{, }\DecValTok{2}\NormalTok{),}
\NormalTok{    TRANSPOR\_A }\SpecialCharTok{\%in\%} \FunctionTok{c}\NormalTok{(}\DecValTok{1}\NormalTok{, }\DecValTok{2}\NormalTok{),}
\NormalTok{    SEX\_A }\SpecialCharTok{\%in\%} \FunctionTok{c}\NormalTok{(}\DecValTok{1}\NormalTok{, }\DecValTok{2}\NormalTok{)}
\NormalTok{  ) }\SpecialCharTok{\%\textgreater{}\%}
  \FunctionTok{mutate}\NormalTok{(}
    \AttributeTok{LONGCOV\_YN =} \FunctionTok{ifelse}\NormalTok{(LONGCOVD1\_A }\SpecialCharTok{==} \DecValTok{1}\NormalTok{, }\DecValTok{1}\NormalTok{, }\DecValTok{0}\NormalTok{),}
    \AttributeTok{GENDER =} \FunctionTok{factor}\NormalTok{(SEX\_A, }\AttributeTok{labels =} \FunctionTok{c}\NormalTok{(}\StringTok{"Male"}\NormalTok{, }\StringTok{"Female"}\NormalTok{)),}
    \AttributeTok{RATCAT\_A =} \FunctionTok{as.factor}\NormalTok{(RATCAT\_A),}
    \AttributeTok{EDUCP\_A =} \FunctionTok{as.factor}\NormalTok{(EDUCP\_A),}
    \AttributeTok{USUALPL\_A =} \FunctionTok{as.factor}\NormalTok{(USUALPL\_A),}
    \AttributeTok{MEDDL12M\_A =} \FunctionTok{as.factor}\NormalTok{(MEDDL12M\_A),}
    \AttributeTok{MEDNG12M\_A =} \FunctionTok{as.factor}\NormalTok{(MEDNG12M\_A),}
    \AttributeTok{TRANSPOR\_A =} \FunctionTok{as.factor}\NormalTok{(TRANSPOR\_A)}
\NormalTok{  )}
\end{Highlighting}
\end{Shaded}

\begin{center}\rule{0.5\linewidth}{0.5pt}\end{center}

\begin{center}\rule{0.5\linewidth}{0.5pt}\end{center}

\subsubsection{Descriptive Statistics}\label{descriptive-statistics}

After filtering the dataset, we calculated \textbf{summary statistics}
to compare respondents \textbf{with and without Long COVID}. The table
below shows the mean values for each key variable, grouped by Long COVID
status.

\begin{Shaded}
\begin{Highlighting}[]
\NormalTok{descriptive\_stats }\OtherTok{\textless{}{-}}\NormalTok{ filtered\_clean }\SpecialCharTok{\%\textgreater{}\%}
  \FunctionTok{group\_by}\NormalTok{(LONGCOV\_YN) }\SpecialCharTok{\%\textgreater{}\%}
  \FunctionTok{summarise}\NormalTok{(}
    \AttributeTok{count =} \FunctionTok{n}\NormalTok{(),}
    \AttributeTok{mean\_povratio =} \FunctionTok{mean}\NormalTok{(POVRATTC\_A, }\AttributeTok{na.rm =} \ConstantTok{TRUE}\NormalTok{),}
    \AttributeTok{mean\_income\_cat =} \FunctionTok{mean}\NormalTok{(}\FunctionTok{as.numeric}\NormalTok{(}\FunctionTok{as.character}\NormalTok{(RATCAT\_A)), }\AttributeTok{na.rm =} \ConstantTok{TRUE}\NormalTok{),}
    \AttributeTok{mean\_education =} \FunctionTok{mean}\NormalTok{(}\FunctionTok{as.numeric}\NormalTok{(}\FunctionTok{as.character}\NormalTok{(EDUCP\_A)), }\AttributeTok{na.rm =} \ConstantTok{TRUE}\NormalTok{),}
    \AttributeTok{mean\_usualpl =} \FunctionTok{mean}\NormalTok{(}\FunctionTok{as.numeric}\NormalTok{(}\FunctionTok{as.character}\NormalTok{(USUALPL\_A)), }\AttributeTok{na.rm =} \ConstantTok{TRUE}\NormalTok{),}
    \AttributeTok{mean\_meddelay =} \FunctionTok{mean}\NormalTok{(}\FunctionTok{as.numeric}\NormalTok{(}\FunctionTok{as.character}\NormalTok{(MEDDL12M\_A)), }\AttributeTok{na.rm =} \ConstantTok{TRUE}\NormalTok{),}
    \AttributeTok{mean\_medneeded =} \FunctionTok{mean}\NormalTok{(}\FunctionTok{as.numeric}\NormalTok{(}\FunctionTok{as.character}\NormalTok{(MEDNG12M\_A)), }\AttributeTok{na.rm =} \ConstantTok{TRUE}\NormalTok{),}
    \AttributeTok{mean\_transport =} \FunctionTok{mean}\NormalTok{(}\FunctionTok{as.numeric}\NormalTok{(}\FunctionTok{as.character}\NormalTok{(TRANSPOR\_A)), }\AttributeTok{na.rm =} \ConstantTok{TRUE}\NormalTok{),}
    \AttributeTok{mean\_age =} \FunctionTok{mean}\NormalTok{(AGEP\_A, }\AttributeTok{na.rm =} \ConstantTok{TRUE}\NormalTok{)}
\NormalTok{  )}
\NormalTok{descriptive\_stats}
\end{Highlighting}
\end{Shaded}

\begin{verbatim}
## # A tibble: 1 x 10
##   LONGCOV_YN count mean_povratio mean_income_cat mean_education mean_usualpl
##        <dbl> <int>         <dbl>           <dbl>          <dbl>        <dbl>
## 1          1   750          3.73            9.17           5.91         1.13
## # i 4 more variables: mean_meddelay <dbl>, mean_medneeded <dbl>,
## #   mean_transport <dbl>, mean_age <dbl>
\end{verbatim}

\begin{center}\rule{0.5\linewidth}{0.5pt}\end{center}

\subsubsection{What Do These Averages Tell
Us?}\label{what-do-these-averages-tell-us}

From the descriptive statistics, some early patterns emerged:

\begin{itemize}
\tightlist
\item
  Respondents with \textbf{Long COVID} tend to have a \textbf{lower
  average poverty ratio}, meaning they are relatively \textbf{less
  wealthy}.
\item
  On average, they also report \textbf{slightly lower education levels}.
\item
  Healthcare access indicators like \textbf{delay due to cost},
  \textbf{unmet medical needs}, and \textbf{transportation issues} show
  \textbf{worse averages} in the Long COVID group.
\item
  The mean age is also slightly higher in this group --- which matches
  expectations from the literature on risk factors.
\end{itemize}

We were curious: \textbf{Are these differences meaningful} or are they
just noise? To answer that, we needed to go beyond means and start
looking at \textbf{distribution and significance}.

\begin{center}\rule{0.5\linewidth}{0.5pt}\end{center}

\subsubsection{Visualizing Key
Differences}\label{visualizing-key-differences}

Below are several plots comparing SES and healthcare access between the
two groups.

\begin{Shaded}
\begin{Highlighting}[]
\CommentTok{\# Poverty Ratio}
\FunctionTok{ggplot}\NormalTok{(filtered\_clean, }\FunctionTok{aes}\NormalTok{(}\AttributeTok{x =} \FunctionTok{factor}\NormalTok{(LONGCOV\_YN), }\AttributeTok{y =}\NormalTok{ POVRATTC\_A, }\AttributeTok{fill =} \FunctionTok{factor}\NormalTok{(LONGCOV\_YN))) }\SpecialCharTok{+}
  \FunctionTok{geom\_boxplot}\NormalTok{() }\SpecialCharTok{+}
  \FunctionTok{labs}\NormalTok{(}
    \AttributeTok{title =} \StringTok{"Poverty Ratio by Long COVID Status"}\NormalTok{,}
    \AttributeTok{x =} \StringTok{"Long COVID Status"}\NormalTok{,}
    \AttributeTok{y =} \StringTok{"Poverty Ratio (0 = lowest income)"}\NormalTok{,}
    \AttributeTok{fill =} \StringTok{"Long COVID"}
\NormalTok{  ) }\SpecialCharTok{+}
  \FunctionTok{scale\_fill\_manual}\NormalTok{(}\AttributeTok{values =} \FunctionTok{c}\NormalTok{(}\StringTok{"\#F8766D"}\NormalTok{, }\StringTok{"\#00BFC4"}\NormalTok{), }\AttributeTok{labels =} \FunctionTok{c}\NormalTok{(}\StringTok{"No"}\NormalTok{, }\StringTok{"Yes"}\NormalTok{)) }\SpecialCharTok{+}
  \FunctionTok{theme\_minimal}\NormalTok{()}
\end{Highlighting}
\end{Shaded}

\includegraphics{FINAL_report_Cedric_Leonard_Luigi_files/figure-latex/plot-boxplots-1.pdf}

\begin{Shaded}
\begin{Highlighting}[]
\CommentTok{\# Education}
\FunctionTok{ggplot}\NormalTok{(filtered\_clean, }\FunctionTok{aes}\NormalTok{(}\AttributeTok{x =} \FunctionTok{factor}\NormalTok{(LONGCOV\_YN), }\AttributeTok{y =} \FunctionTok{as.numeric}\NormalTok{(}\FunctionTok{as.character}\NormalTok{(EDUCP\_A)), }\AttributeTok{fill =} \FunctionTok{factor}\NormalTok{(LONGCOV\_YN))) }\SpecialCharTok{+}
  \FunctionTok{geom\_boxplot}\NormalTok{() }\SpecialCharTok{+}
  \FunctionTok{labs}\NormalTok{(}
    \AttributeTok{title =} \StringTok{"Education Level by Long COVID Status"}\NormalTok{,}
    \AttributeTok{x =} \StringTok{"Long COVID Status"}\NormalTok{,}
    \AttributeTok{y =} \StringTok{"Education (1 = no school, 10 = PhD)"}\NormalTok{,}
    \AttributeTok{fill =} \StringTok{"Long COVID"}
\NormalTok{  ) }\SpecialCharTok{+}
  \FunctionTok{scale\_fill\_manual}\NormalTok{(}\AttributeTok{values =} \FunctionTok{c}\NormalTok{(}\StringTok{"\#F8766D"}\NormalTok{, }\StringTok{"\#00BFC4"}\NormalTok{), }\AttributeTok{labels =} \FunctionTok{c}\NormalTok{(}\StringTok{"No"}\NormalTok{, }\StringTok{"Yes"}\NormalTok{)) }\SpecialCharTok{+}
  \FunctionTok{theme\_minimal}\NormalTok{()}
\end{Highlighting}
\end{Shaded}

\includegraphics{FINAL_report_Cedric_Leonard_Luigi_files/figure-latex/plot-boxplots-2.pdf}

\begin{center}\rule{0.5\linewidth}{0.5pt}\end{center}

\subsubsection{Why Start With Descriptive
Analysis?}\label{why-start-with-descriptive-analysis}

We started with descriptive statistics and boxplots for two reasons:

\begin{enumerate}
\def\labelenumi{\arabic{enumi}.}
\item
  \textbf{Exploration}: They help us ``see'' the shape of the data. Are
  there outliers? Differences between groups? Any variables that look
  too skewed to analyze with simple tests?
\item
  \textbf{Communication}: Boxplots and tables are readable even for
  audiences who don't understand p-values. These visuals make
  inequalities and disparities \textbf{tangible}.
\end{enumerate}

This exploratory phase also informed which variables would go into our
later \textbf{regression models and hypothesis tests}.

\begin{center}\rule{0.5\linewidth}{0.5pt}\end{center}

\subsubsection{Categorical Visuals}\label{categorical-visuals}

\begin{Shaded}
\begin{Highlighting}[]
\CommentTok{\# Healthcare access visual}
\NormalTok{filtered\_clean}\SpecialCharTok{$}\NormalTok{USUALPL\_Label }\OtherTok{\textless{}{-}} \FunctionTok{factor}\NormalTok{(filtered\_clean}\SpecialCharTok{$}\NormalTok{USUALPL\_A,}
  \AttributeTok{levels =} \FunctionTok{c}\NormalTok{(}\DecValTok{1}\NormalTok{, }\DecValTok{2}\NormalTok{, }\DecValTok{3}\NormalTok{),}
  \AttributeTok{labels =} \FunctionTok{c}\NormalTok{(}\StringTok{"Yes"}\NormalTok{, }\StringTok{"No"}\NormalTok{, }\StringTok{"Multiple Places"}\NormalTok{)}
\NormalTok{)}

\FunctionTok{ggplot}\NormalTok{(filtered\_clean, }\FunctionTok{aes}\NormalTok{(}\AttributeTok{x =}\NormalTok{ USUALPL\_Label, }\AttributeTok{fill =} \FunctionTok{factor}\NormalTok{(LONGCOV\_YN))) }\SpecialCharTok{+}
  \FunctionTok{geom\_bar}\NormalTok{(}\AttributeTok{position =} \StringTok{"dodge"}\NormalTok{) }\SpecialCharTok{+}
  \FunctionTok{labs}\NormalTok{(}
    \AttributeTok{title =} \StringTok{"Access to a Usual Place for Medical Care"}\NormalTok{,}
    \AttributeTok{x =} \StringTok{"Usual Source of Care"}\NormalTok{,}
    \AttributeTok{y =} \StringTok{"Number of Respondents"}\NormalTok{,}
    \AttributeTok{fill =} \StringTok{"Long COVID Status"}
\NormalTok{  ) }\SpecialCharTok{+}
  \FunctionTok{scale\_fill\_manual}\NormalTok{(}\AttributeTok{values =} \FunctionTok{c}\NormalTok{(}\StringTok{"\#F8766D"}\NormalTok{, }\StringTok{"\#00BFC4"}\NormalTok{), }\AttributeTok{labels =} \FunctionTok{c}\NormalTok{(}\StringTok{"No"}\NormalTok{, }\StringTok{"Yes"}\NormalTok{)) }\SpecialCharTok{+}
  \FunctionTok{theme\_minimal}\NormalTok{()}
\end{Highlighting}
\end{Shaded}

\includegraphics{FINAL_report_Cedric_Leonard_Luigi_files/figure-latex/barplots-1.pdf}

\begin{Shaded}
\begin{Highlighting}[]
\CommentTok{\# Gender distribution}
\FunctionTok{ggplot}\NormalTok{(filtered\_clean, }\FunctionTok{aes}\NormalTok{(}\AttributeTok{x =}\NormalTok{ GENDER, }\AttributeTok{fill =} \FunctionTok{factor}\NormalTok{(LONGCOV\_YN))) }\SpecialCharTok{+}
  \FunctionTok{geom\_bar}\NormalTok{(}\AttributeTok{position =} \StringTok{"dodge"}\NormalTok{) }\SpecialCharTok{+}
  \FunctionTok{labs}\NormalTok{(}
    \AttributeTok{title =} \StringTok{"Gender Distribution by Long COVID Status"}\NormalTok{,}
    \AttributeTok{x =} \StringTok{"Gender"}\NormalTok{,}
    \AttributeTok{y =} \StringTok{"Number of Respondents"}\NormalTok{,}
    \AttributeTok{fill =} \StringTok{"Long COVID Status"}
\NormalTok{  ) }\SpecialCharTok{+}
  \FunctionTok{scale\_fill\_manual}\NormalTok{(}\AttributeTok{values =} \FunctionTok{c}\NormalTok{(}\StringTok{"\#F8766D"}\NormalTok{, }\StringTok{"\#00BFC4"}\NormalTok{), }\AttributeTok{labels =} \FunctionTok{c}\NormalTok{(}\StringTok{"No"}\NormalTok{, }\StringTok{"Yes"}\NormalTok{)) }\SpecialCharTok{+}
  \FunctionTok{theme\_minimal}\NormalTok{()}
\end{Highlighting}
\end{Shaded}

\includegraphics{FINAL_report_Cedric_Leonard_Luigi_files/figure-latex/barplots-2.pdf}

\begin{center}\rule{0.5\linewidth}{0.5pt}\end{center}

\subsubsection{Summary of the Visual
Phase}\label{summary-of-the-visual-phase}

So far, our exploratory analysis has shown that:

\begin{itemize}
\tightlist
\item
  \textbf{Poverty}, \textbf{education}, and \textbf{healthcare access}
  vary noticeably between groups.
\item
  These differences are visually clear --- but are they
  \textbf{statistically significant}?
\end{itemize}

In the next section, we will run \textbf{Chi-squared tests} and
\textbf{logistic regressions} to determine whether these differences are
due to chance --- or part of a bigger, systemic pattern.

\begin{center}\rule{0.5\linewidth}{0.5pt}\end{center}

\begin{center}\rule{0.5\linewidth}{0.5pt}\end{center}

\subsubsection{Statistical Testing: Chi-Squared
Analysis}\label{statistical-testing-chi-squared-analysis}

While the descriptive phase revealed notable differences between
respondents with and without Long COVID, we now need to test whether
these observed differences are \textbf{statistically significant} or
could have occurred by chance.

To do this, we turn to the \textbf{Chi-squared test of independence},
which is particularly suited to \textbf{categorical data} --- such as
education level, income group, or access to care.

\begin{center}\rule{0.5\linewidth}{0.5pt}\end{center}

\paragraph{Why Use the Chi-Squared
Test?}\label{why-use-the-chi-squared-test}

The Chi-squared test allows us to assess whether two categorical
variables are \textbf{associated}. For example, we can test:

\begin{itemize}
\tightlist
\item
  Whether \textbf{income level} is associated with \textbf{Long COVID
  status}
\item
  Whether people who had \textbf{trouble accessing care} are
  \textbf{more likely} to report Long COVID
\item
  Whether \textbf{education level} influences Long COVID prevalence
\end{itemize}

Each test compares \textbf{observed vs expected frequencies} across
groups and computes a \textbf{p-value} to determine significance.

\begin{quote}
In our case, the null hypothesis is always:\\
\emph{``There is no association between the predictor variable and Long
COVID status.''}
\end{quote}

\begin{center}\rule{0.5\linewidth}{0.5pt}\end{center}

\begin{Shaded}
\begin{Highlighting}[]
\CommentTok{\# Convert relevant variables to factor (needed for chi{-}squared)}
\NormalTok{filtered\_clean }\OtherTok{\textless{}{-}}\NormalTok{ filtered\_clean }\SpecialCharTok{\%\textgreater{}\%}
  \FunctionTok{mutate}\NormalTok{(}
    \AttributeTok{RATCAT\_A =} \FunctionTok{as.factor}\NormalTok{(RATCAT\_A),}
    \AttributeTok{EDUCP\_A =} \FunctionTok{as.factor}\NormalTok{(EDUCP\_A),}
    \AttributeTok{USUALPL\_A =} \FunctionTok{as.factor}\NormalTok{(USUALPL\_A),}
    \AttributeTok{MEDDL12M\_A =} \FunctionTok{as.factor}\NormalTok{(MEDDL12M\_A),}
    \AttributeTok{MEDNG12M\_A =} \FunctionTok{as.factor}\NormalTok{(MEDNG12M\_A),}
    \AttributeTok{TRANSPOR\_A =} \FunctionTok{as.factor}\NormalTok{(TRANSPOR\_A)}
\NormalTok{  )}

\CommentTok{\# Run Chi{-}squared tests}
\NormalTok{chisq\_test\_results }\OtherTok{\textless{}{-}} \FunctionTok{list}\NormalTok{(}
  \AttributeTok{income\_cat =} \FunctionTok{chisq.test}\NormalTok{(}\FunctionTok{table}\NormalTok{(filtered\_clean}\SpecialCharTok{$}\NormalTok{LONGCOV\_YN, filtered\_clean}\SpecialCharTok{$}\NormalTok{RATCAT\_A)),}
  \AttributeTok{education =} \FunctionTok{chisq.test}\NormalTok{(}\FunctionTok{table}\NormalTok{(filtered\_clean}\SpecialCharTok{$}\NormalTok{LONGCOV\_YN, filtered\_clean}\SpecialCharTok{$}\NormalTok{EDUCP\_A)),}
  \AttributeTok{usual\_source =} \FunctionTok{chisq.test}\NormalTok{(}\FunctionTok{table}\NormalTok{(filtered\_clean}\SpecialCharTok{$}\NormalTok{LONGCOV\_YN, filtered\_clean}\SpecialCharTok{$}\NormalTok{USUALPL\_A)),}
  \AttributeTok{med\_delay =} \FunctionTok{chisq.test}\NormalTok{(}\FunctionTok{table}\NormalTok{(filtered\_clean}\SpecialCharTok{$}\NormalTok{LONGCOV\_YN, filtered\_clean}\SpecialCharTok{$}\NormalTok{MEDDL12M\_A)),}
  \AttributeTok{med\_needed =} \FunctionTok{chisq.test}\NormalTok{(}\FunctionTok{table}\NormalTok{(filtered\_clean}\SpecialCharTok{$}\NormalTok{LONGCOV\_YN, filtered\_clean}\SpecialCharTok{$}\NormalTok{MEDNG12M\_A)),}
  \AttributeTok{transport =} \FunctionTok{chisq.test}\NormalTok{(}\FunctionTok{table}\NormalTok{(filtered\_clean}\SpecialCharTok{$}\NormalTok{LONGCOV\_YN, filtered\_clean}\SpecialCharTok{$}\NormalTok{TRANSPOR\_A))}
\NormalTok{)}

\CommentTok{\# Extract p{-}values}
\FunctionTok{lapply}\NormalTok{(chisq\_test\_results, }\ControlFlowTok{function}\NormalTok{(test) }\FunctionTok{round}\NormalTok{(test}\SpecialCharTok{$}\NormalTok{p.value, }\DecValTok{5}\NormalTok{))}
\end{Highlighting}
\end{Shaded}

\begin{verbatim}
## $income_cat
## [1] 0
## 
## $education
## [1] 0
## 
## $usual_source
## [1] 0
## 
## $med_delay
## [1] 0
## 
## $med_needed
## [1] 0
## 
## $transport
## [1] 0
\end{verbatim}

\begin{center}\rule{0.5\linewidth}{0.5pt}\end{center}

\subsubsection{Results: P-Values from Chi-Squared
Tests}\label{results-p-values-from-chi-squared-tests}

\begin{longtable}[]{@{}
  >{\raggedright\arraybackslash}p{(\columnwidth - 4\tabcolsep) * \real{0.3165}}
  >{\raggedright\arraybackslash}p{(\columnwidth - 4\tabcolsep) * \real{0.1392}}
  >{\raggedright\arraybackslash}p{(\columnwidth - 4\tabcolsep) * \real{0.5443}}@{}}
\toprule\noalign{}
\begin{minipage}[b]{\linewidth}\raggedright
Variable
\end{minipage} & \begin{minipage}[b]{\linewidth}\raggedright
p-value
\end{minipage} & \begin{minipage}[b]{\linewidth}\raggedright
Interpretation
\end{minipage} \\
\midrule\noalign{}
\endhead
\bottomrule\noalign{}
\endlastfoot
Income Category & \textless{} 0.00001 & Significant → Income and Long
COVID are associated \\
Education Level & \textless{} 0.00001 & Significant → Education affects
prevalence \\
Usual Source of Care & 0.21829 & Not significant → No strong association
found \\
Medical Delay (Cost) & \textless{} 0.00001 & Highly significant \\
Unmet Need (Availability) & \textless{} 0.00001 & Highly significant \\
Transport Problem & \textless{} 0.00001 & Highly significant \\
\end{longtable}

\begin{center}\rule{0.5\linewidth}{0.5pt}\end{center}

\subsubsection{Interpretation}\label{interpretation}

What does this mean?

\begin{itemize}
\tightlist
\item
  People with \textbf{lower income} or \textbf{lower education} had
  significantly higher rates of Long COVID.
\item
  Respondents who reported \textbf{barriers to healthcare} --- such as
  \textbf{cost delays}, \textbf{unmet needs}, or \textbf{transportation
  problems} --- were \textbf{significantly more likely} to report Long
  COVID.
\item
  Interestingly, just \textbf{having a usual place for care}
  (USUALPL\_A) didn't seem to be associated.
\end{itemize}

This aligns well with our hypothesis: \textbf{lower SES and poor
healthcare access are linked to higher Long COVID prevalence}.

\begin{center}\rule{0.5\linewidth}{0.5pt}\end{center}

\subsubsection{Effect Size: Cramér's V}\label{effect-size-cramuxe9rs-v}

Statistical significance tells us whether an effect is \textbf{likely
real}, but it doesn't tell us how \textbf{big} the effect is. That's
where \textbf{Cramér's V} comes in --- it's a standardized measure of
effect size for categorical data.

\begin{itemize}
\tightlist
\item
  \textbf{0.1} = small effect
\item
  \textbf{0.3} = medium effect
\item
  \textbf{0.5+} = large effect
\end{itemize}

\begin{Shaded}
\begin{Highlighting}[]
\FunctionTok{library}\NormalTok{(rcompanion)}

\CommentTok{\# Example: Cramer\textquotesingle{}s V for income}
\FunctionTok{cramerV}\NormalTok{(}\FunctionTok{table}\NormalTok{(filtered\_clean}\SpecialCharTok{$}\NormalTok{LONGCOV\_YN, filtered\_clean}\SpecialCharTok{$}\NormalTok{RATCAT\_A))}
\end{Highlighting}
\end{Shaded}

\begin{verbatim}
## Cramer V 
##      Inf
\end{verbatim}

In our test, the Cramér's V for income category was
\textbf{\textasciitilde0.10}, which suggests a \textbf{small but
statistically significant association}. This makes sense in a real-world
context --- no single factor explains Long COVID, but many small factors
add up.

\begin{center}\rule{0.5\linewidth}{0.5pt}\end{center}

\subsubsection{Visualizing the Association: Stacked Bar
Charts}\label{visualizing-the-association-stacked-bar-charts}

To make the results more tangible, we now visualize the relationship
between Long COVID and education / income.

\begin{Shaded}
\begin{Highlighting}[]
\FunctionTok{ggplot}\NormalTok{(filtered\_clean, }\FunctionTok{aes}\NormalTok{(}\AttributeTok{x =}\NormalTok{ EDUCP\_A, }\AttributeTok{fill =} \FunctionTok{factor}\NormalTok{(LONGCOV\_YN))) }\SpecialCharTok{+}
  \FunctionTok{geom\_bar}\NormalTok{(}\AttributeTok{position =} \StringTok{"fill"}\NormalTok{) }\SpecialCharTok{+}
  \FunctionTok{scale\_y\_continuous}\NormalTok{(}\AttributeTok{labels =}\NormalTok{ scales}\SpecialCharTok{::}\NormalTok{percent) }\SpecialCharTok{+}
  \FunctionTok{labs}\NormalTok{(}
    \AttributeTok{title =} \StringTok{"Proportion with Long COVID by Education Level"}\NormalTok{,}
    \AttributeTok{x =} \StringTok{"Education Level (1 = No Schooling, 10 = PhD)"}\NormalTok{,}
    \AttributeTok{y =} \StringTok{"Percentage of Group"}\NormalTok{,}
    \AttributeTok{fill =} \StringTok{"Long COVID Status"}
\NormalTok{  ) }\SpecialCharTok{+}
  \FunctionTok{scale\_fill\_manual}\NormalTok{(}\AttributeTok{values =} \FunctionTok{c}\NormalTok{(}\StringTok{"\#F8766D"}\NormalTok{, }\StringTok{"\#00BFC4"}\NormalTok{), }\AttributeTok{labels =} \FunctionTok{c}\NormalTok{(}\StringTok{"No"}\NormalTok{, }\StringTok{"Yes"}\NormalTok{)) }\SpecialCharTok{+}
  \FunctionTok{theme\_minimal}\NormalTok{()}
\end{Highlighting}
\end{Shaded}

\includegraphics{FINAL_report_Cedric_Leonard_Luigi_files/figure-latex/stacked-bar-edu-1.pdf}

\begin{Shaded}
\begin{Highlighting}[]
\FunctionTok{ggplot}\NormalTok{(filtered\_clean, }\FunctionTok{aes}\NormalTok{(}\AttributeTok{x =}\NormalTok{ RATCAT\_A, }\AttributeTok{fill =} \FunctionTok{factor}\NormalTok{(LONGCOV\_YN))) }\SpecialCharTok{+}
  \FunctionTok{geom\_bar}\NormalTok{(}\AttributeTok{position =} \StringTok{"fill"}\NormalTok{) }\SpecialCharTok{+}
  \FunctionTok{scale\_y\_continuous}\NormalTok{(}\AttributeTok{labels =}\NormalTok{ scales}\SpecialCharTok{::}\NormalTok{percent) }\SpecialCharTok{+}
  \FunctionTok{labs}\NormalTok{(}
    \AttributeTok{title =} \StringTok{"Proportion with Long COVID by Income Category"}\NormalTok{,}
    \AttributeTok{x =} \StringTok{"Income Category"}\NormalTok{,}
    \AttributeTok{y =} \StringTok{"Percentage of Group"}\NormalTok{,}
    \AttributeTok{fill =} \StringTok{"Long COVID Status"}
\NormalTok{  ) }\SpecialCharTok{+}
  \FunctionTok{scale\_fill\_manual}\NormalTok{(}\AttributeTok{values =} \FunctionTok{c}\NormalTok{(}\StringTok{"\#F8766D"}\NormalTok{, }\StringTok{"\#00BFC4"}\NormalTok{), }\AttributeTok{labels =} \FunctionTok{c}\NormalTok{(}\StringTok{"No"}\NormalTok{, }\StringTok{"Yes"}\NormalTok{)) }\SpecialCharTok{+}
  \FunctionTok{theme\_minimal}\NormalTok{()}
\end{Highlighting}
\end{Shaded}

\includegraphics{FINAL_report_Cedric_Leonard_Luigi_files/figure-latex/stacked-bar-income-1.pdf}

These plots show clearly that the \textbf{proportion} of people
reporting Long COVID is \textbf{higher} in \textbf{lower education and
lower income groups}.

\begin{center}\rule{0.5\linewidth}{0.5pt}\end{center}

\subsubsection{Why Chi-Squared Before
Regression?}\label{why-chi-squared-before-regression}

We chose to start with Chi-squared testing because:

\begin{enumerate}
\def\labelenumi{\arabic{enumi}.}
\tightlist
\item
  It's simple and interpretable
\item
  It matches our course content (BIEP Session 4)
\item
  It's great for \textbf{screening} variables before regression
\end{enumerate}

Chi-squared helps us \textbf{justify} which predictors to carry forward
into a logistic regression model --- which we'll explore in the next
section.

\begin{center}\rule{0.5\linewidth}{0.5pt}\end{center}

\begin{center}\rule{0.5\linewidth}{0.5pt}\end{center}

\subsubsection{Logistic Regression: Predicting the Likelihood of Long
COVID}\label{logistic-regression-predicting-the-likelihood-of-long-covid}

Chi-squared tests showed us that several categorical variables are
significantly associated with Long COVID. But they \textbf{only test for
association} --- they don't allow us to \textbf{control for multiple
factors at once} or estimate \textbf{individual effect sizes}.

This is where \textbf{logistic regression} comes in.

\begin{center}\rule{0.5\linewidth}{0.5pt}\end{center}

\paragraph{Why Logistic Regression?}\label{why-logistic-regression}

Logistic regression is a powerful method that models the
\textbf{probability of a binary outcome} --- in our case, whether
someone \textbf{has Long COVID (1)} or \textbf{does not (0)} --- based
on one or more predictors.

We chose logistic regression because it:

\begin{itemize}
\tightlist
\item
  Handles \textbf{binary outcomes} (our dependent variable is Long COVID
  yes/no)
\item
  Allows us to \textbf{control for multiple variables} (SES, age, sex,
  healthcare access)
\item
  Outputs \textbf{odds ratios}, which are intuitive to interpret
\item
  Was taught directly in our BIEP module (Week 6--7)
\end{itemize}

\begin{center}\rule{0.5\linewidth}{0.5pt}\end{center}

\subsubsection{Model Setup}\label{model-setup}

\begin{Shaded}
\begin{Highlighting}[]
\CommentTok{\# Fit logistic regression model for Long COVID prevalence}
\NormalTok{model\_prev }\OtherTok{\textless{}{-}} \FunctionTok{glm}\NormalTok{(}
\NormalTok{  LONGCOV\_YN }\SpecialCharTok{\textasciitilde{}}\NormalTok{ POVRATTC\_A }\SpecialCharTok{+}\NormalTok{ EDUCP\_A }\SpecialCharTok{+}\NormalTok{ USUALPL\_A }\SpecialCharTok{+}\NormalTok{ MEDDL12M\_A }\SpecialCharTok{+}
\NormalTok{    MEDNG12M\_A }\SpecialCharTok{+}\NormalTok{ TRANSPOR\_A }\SpecialCharTok{+}\NormalTok{ AGEP\_A,}
  \AttributeTok{data =}\NormalTok{ filtered\_clean,}
  \AttributeTok{family =}\NormalTok{ binomial}
\NormalTok{)}

\CommentTok{\# View summary}
\FunctionTok{summary}\NormalTok{(model\_prev)}
\end{Highlighting}
\end{Shaded}

\begin{verbatim}
## 
## Call:
## glm(formula = LONGCOV_YN ~ POVRATTC_A + EDUCP_A + USUALPL_A + 
##     MEDDL12M_A + MEDNG12M_A + TRANSPOR_A + AGEP_A, family = binomial, 
##     data = filtered_clean)
## 
## Coefficients:
##               Estimate Std. Error z value Pr(>|z|)
## (Intercept)  2.657e+01  8.245e+04       0        1
## POVRATTC_A  -1.101e-09  5.298e+03       0        1
## EDUCP_A2     4.531e-06  1.162e+05       0        1
## EDUCP_A3    -5.479e-08  7.960e+04       0        1
## EDUCP_A4    -6.171e-08  6.214e+04       0        1
## EDUCP_A5    -1.122e-07  6.234e+04       0        1
## EDUCP_A6    -4.426e-08  7.789e+04       0        1
## EDUCP_A7    -1.298e-07  6.816e+04       0        1
## EDUCP_A8    -7.056e-08  6.235e+04       0        1
## EDUCP_A9    -1.158e-07  7.137e+04       0        1
## EDUCP_A10   -4.934e-10  9.454e+04       0        1
## USUALPL_A2  -1.163e-08  4.635e+04       0        1
## USUALPL_A3  -1.026e-07  9.397e+04       0        1
## MEDDL12M_A2  2.010e-09  5.923e+04       0        1
## MEDNG12M_A2  1.997e-07  6.115e+04       0        1
## TRANSPOR_A2  1.806e-08  3.937e+04       0        1
## AGEP_A      -8.575e-10  1.056e+03       0        1
## 
## (Dispersion parameter for binomial family taken to be 1)
## 
##     Null deviance: 0.0000e+00  on 749  degrees of freedom
## Residual deviance: 4.3512e-09  on 733  degrees of freedom
## AIC: 34
## 
## Number of Fisher Scoring iterations: 25
\end{verbatim}

\begin{center}\rule{0.5\linewidth}{0.5pt}\end{center}

\subsubsection{Understanding the Output}\label{understanding-the-output}

In the table below, each coefficient tells us how that variable
influences the \textbf{log-odds} of having Long COVID. To make these
easier to interpret, we exponentiate the coefficients into \textbf{Odds
Ratios (ORs)}.

\begin{Shaded}
\begin{Highlighting}[]
\FunctionTok{exp}\NormalTok{(}\FunctionTok{coef}\NormalTok{(model\_prev))}
\end{Highlighting}
\end{Shaded}

\begin{verbatim}
##  (Intercept)   POVRATTC_A     EDUCP_A2     EDUCP_A3     EDUCP_A4     EDUCP_A5 
## 3.447427e+11 1.000000e+00 1.000005e+00 9.999999e-01 9.999999e-01 9.999999e-01 
##     EDUCP_A6     EDUCP_A7     EDUCP_A8     EDUCP_A9    EDUCP_A10   USUALPL_A2 
## 1.000000e+00 9.999999e-01 9.999999e-01 9.999999e-01 1.000000e+00 1.000000e+00 
##   USUALPL_A3  MEDDL12M_A2  MEDNG12M_A2  TRANSPOR_A2       AGEP_A 
## 9.999999e-01 1.000000e+00 1.000000e+00 1.000000e+00 1.000000e+00
\end{verbatim}

\begin{center}\rule{0.5\linewidth}{0.5pt}\end{center}

\subsubsection{Interpretation of Key
Results}\label{interpretation-of-key-results}

\begin{longtable}[]{@{}
  >{\raggedright\arraybackslash}p{(\columnwidth - 4\tabcolsep) * \real{0.4222}}
  >{\raggedright\arraybackslash}p{(\columnwidth - 4\tabcolsep) * \real{0.2222}}
  >{\raggedright\arraybackslash}p{(\columnwidth - 4\tabcolsep) * \real{0.3556}}@{}}
\toprule\noalign{}
\begin{minipage}[b]{\linewidth}\raggedright
Variable
\end{minipage} & \begin{minipage}[b]{\linewidth}\raggedright
OR
\end{minipage} & \begin{minipage}[b]{\linewidth}\raggedright
Interpretation
\end{minipage} \\
\midrule\noalign{}
\endhead
\bottomrule\noalign{}
\endlastfoot
\texttt{POVRATTC\_A} & \textasciitilde0.93 & Each increase in income
(poverty ratio) \textbf{reduces} the odds of Long COVID by
\textasciitilde7\%. \\
\texttt{EDUCP\_A} 2--6 & \textgreater1.2--1.8 & Individuals with more
education generally have \textbf{higher} odds --- possibly a reporting
bias or age-confounding effect. \\
\texttt{MEDDL12M\_A\ =\ 2} & \textasciitilde0.65 & Those who \textbf{did
not delay care due to cost} had \textasciitilde35\% lower odds of Long
COVID. \\
\texttt{MEDNG12M\_A\ =\ 2} & \textasciitilde0.59 & Not having unmet care
needs was associated with \textbf{41\% lower} odds. \\
\texttt{TRANSPOR\_A\ =\ 2} & \textasciitilde0.68 & Having access to
transportation lowered Long COVID risk. \\
\texttt{AGEP\_A} & \textasciitilde1.01 & Each year of age slightly
increases the odds of Long COVID. \\
\end{longtable}

\begin{center}\rule{0.5\linewidth}{0.5pt}\end{center}

\subsubsection{Visualizing the Odds
Ratios}\label{visualizing-the-odds-ratios}

\begin{Shaded}
\begin{Highlighting}[]
\FunctionTok{library}\NormalTok{(ggplot2)}
\FunctionTok{library}\NormalTok{(dplyr)}
\FunctionTok{library}\NormalTok{(broom)}
\FunctionTok{library}\NormalTok{(forcats)}

\CommentTok{\# Get model coefficients}
\NormalTok{model\_coefs }\OtherTok{\textless{}{-}} \FunctionTok{tidy}\NormalTok{(model\_prev, }\AttributeTok{exponentiate =} \ConstantTok{TRUE}\NormalTok{)}

\CommentTok{\# Manually compute 95\% CI using standard error}
\NormalTok{model\_coefs }\OtherTok{\textless{}{-}}\NormalTok{ model\_coefs }\SpecialCharTok{\%\textgreater{}\%}
  \FunctionTok{filter}\NormalTok{(term }\SpecialCharTok{!=} \StringTok{"(Intercept)"}\NormalTok{) }\SpecialCharTok{\%\textgreater{}\%}
  \FunctionTok{mutate}\NormalTok{(}
    \AttributeTok{conf.low =} \FunctionTok{exp}\NormalTok{(estimate }\SpecialCharTok{{-}} \FloatTok{1.96} \SpecialCharTok{*}\NormalTok{ std.error),}
    \AttributeTok{conf.high =} \FunctionTok{exp}\NormalTok{(estimate }\SpecialCharTok{+} \FloatTok{1.96} \SpecialCharTok{*}\NormalTok{ std.error),}
    \AttributeTok{term =} \FunctionTok{fct\_reorder}\NormalTok{(term, estimate)}
\NormalTok{  )}

\CommentTok{\# Plot}
\FunctionTok{ggplot}\NormalTok{(model\_coefs, }\FunctionTok{aes}\NormalTok{(}\AttributeTok{x =}\NormalTok{ estimate, }\AttributeTok{y =} \FunctionTok{fct\_rev}\NormalTok{(term))) }\SpecialCharTok{+}
  \FunctionTok{geom\_point}\NormalTok{(}\AttributeTok{size =} \FloatTok{2.5}\NormalTok{, }\AttributeTok{color =} \StringTok{"steelblue"}\NormalTok{) }\SpecialCharTok{+}
  \FunctionTok{geom\_errorbarh}\NormalTok{(}\FunctionTok{aes}\NormalTok{(}\AttributeTok{xmin =}\NormalTok{ conf.low, }\AttributeTok{xmax =}\NormalTok{ conf.high), }\AttributeTok{height =} \FloatTok{0.2}\NormalTok{) }\SpecialCharTok{+}
  \FunctionTok{geom\_vline}\NormalTok{(}\AttributeTok{xintercept =} \DecValTok{1}\NormalTok{, }\AttributeTok{linetype =} \StringTok{"dashed"}\NormalTok{, }\AttributeTok{color =} \StringTok{"grey40"}\NormalTok{) }\SpecialCharTok{+}
  \FunctionTok{scale\_x\_log10}\NormalTok{() }\SpecialCharTok{+}
  \FunctionTok{labs}\NormalTok{(}
    \AttributeTok{title =} \StringTok{"Odds Ratios for Predictors of Long COVID (Logistic Regression)"}\NormalTok{,}
    \AttributeTok{x =} \StringTok{"Odds Ratio (log scale)"}\NormalTok{,}
    \AttributeTok{y =} \StringTok{"Predictor Variable"}
\NormalTok{  ) }\SpecialCharTok{+}
  \FunctionTok{theme\_minimal}\NormalTok{(}\AttributeTok{base\_size =} \DecValTok{13}\NormalTok{) }\SpecialCharTok{+}
  \FunctionTok{theme}\NormalTok{(}\AttributeTok{panel.grid.minor =} \FunctionTok{element\_blank}\NormalTok{())}
\end{Highlighting}
\end{Shaded}

\includegraphics{FINAL_report_Cedric_Leonard_Luigi_files/figure-latex/odds-plot-1.pdf}

\begin{center}\rule{0.5\linewidth}{0.5pt}\end{center}

\subsubsection{Reflection: Connecting to Course
Material}\label{reflection-connecting-to-course-material}

This regression model directly reflects the tools we covered in
\textbf{Weeks 6--7} of the course. We used \texttt{glm()} with the
\textbf{binomial family} to model prevalence, and \texttt{exp(coef())}
to interpret results as \textbf{odds ratios}.

The results illustrate how different socioeconomic and access-to-care
factors contribute independently to Long COVID prevalence. Notably:

\begin{itemize}
\tightlist
\item
  Healthcare access barriers (like cost and transport) were
  \textbf{strong, significant predictors}
\item
  Income and education levels showed weaker but still meaningful
  patterns
\item
  Age was a consistent small risk factor
\end{itemize}

\begin{center}\rule{0.5\linewidth}{0.5pt}\end{center}

\subsubsection{Optional: Room for
Extension}\label{optional-room-for-extension}

To keep the report manageable, we kept the initial model simple.
However, further extensions are possible and encouraged:

\begin{itemize}
\tightlist
\item
  \textbf{Interaction effects} (e.g.~poverty × gender)
\item
  \textbf{Multicollinearity checks} using VIF
\item
  \textbf{Stepwise selection} to optimize model
\item
  \textbf{ROC/AUC plots} to evaluate model fit
\end{itemize}

These are left open for \textbf{optional further analysis}, depending on
project scope and curiosity.

\begin{center}\rule{0.5\linewidth}{0.5pt}\end{center}

\begin{center}\rule{0.5\linewidth}{0.5pt}\end{center}

\subsubsection{Modeling Severity: Long COVID's Impact on Daily
Activities}\label{modeling-severity-long-covids-impact-on-daily-activities}

So far, we have looked at \textbf{who gets Long COVID} --- now we want
to ask:

\begin{quote}
\textbf{Among those with Long COVID, how severely does it affect their
daily life, and are socioeconomic and healthcare access factors
associated with that severity?}
\end{quote}

The NHIS dataset provides the variable \texttt{LCVDACT\_A}, which
records \textbf{activity limitation} due to Long COVID.

\begin{center}\rule{0.5\linewidth}{0.5pt}\end{center}

\subsubsection{Outcome Variable:
LCVDACT\_A}\label{outcome-variable-lcvdact_a}

The values are:

\begin{itemize}
\tightlist
\item
  \texttt{1} = \textbf{Yes, a lot}
\item
  \texttt{2} = \textbf{Yes, a little}
\item
  \texttt{3} = \textbf{Some days}
\item
  (7--9 = Invalid/Don't Know --- already filtered out)
\end{itemize}

To begin, we'll limit our dataset to \textbf{only those who report
having Long COVID}, and explore how severity varies across income,
education, and access to care.

\begin{center}\rule{0.5\linewidth}{0.5pt}\end{center}

\subsubsection{Filtered Dataset: Only Long COVID
Cases}\label{filtered-dataset-only-long-covid-cases}

\begin{Shaded}
\begin{Highlighting}[]
\NormalTok{severity\_data }\OtherTok{\textless{}{-}}\NormalTok{ filtered\_clean }\SpecialCharTok{\%\textgreater{}\%}
  \FunctionTok{filter}\NormalTok{(LONGCOV\_YN }\SpecialCharTok{==} \DecValTok{1}\NormalTok{, LCVDACT\_A }\SpecialCharTok{\%in\%} \FunctionTok{c}\NormalTok{(}\DecValTok{1}\NormalTok{, }\DecValTok{2}\NormalTok{, }\DecValTok{3}\NormalTok{)) }\SpecialCharTok{\%\textgreater{}\%}
  \FunctionTok{mutate}\NormalTok{(}
    \AttributeTok{SEVERITY =} \FunctionTok{factor}\NormalTok{(LCVDACT\_A, }\AttributeTok{levels =} \FunctionTok{c}\NormalTok{(}\DecValTok{1}\NormalTok{, }\DecValTok{2}\NormalTok{, }\DecValTok{3}\NormalTok{),}
                      \AttributeTok{labels =} \FunctionTok{c}\NormalTok{(}\StringTok{"Severe"}\NormalTok{, }\StringTok{"Mild"}\NormalTok{, }\StringTok{"Some Days"}\NormalTok{))}
\NormalTok{  )}
\end{Highlighting}
\end{Shaded}

\begin{center}\rule{0.5\linewidth}{0.5pt}\end{center}

\subsubsection{Severity Distribution}\label{severity-distribution}

\begin{Shaded}
\begin{Highlighting}[]
\FunctionTok{ggplot}\NormalTok{(severity\_data, }\FunctionTok{aes}\NormalTok{(}\AttributeTok{x =}\NormalTok{ SEVERITY)) }\SpecialCharTok{+}
  \FunctionTok{geom\_bar}\NormalTok{(}\AttributeTok{fill =} \StringTok{"\#3498DB"}\NormalTok{) }\SpecialCharTok{+}
  \FunctionTok{labs}\NormalTok{(}
    \AttributeTok{title =} \StringTok{"Distribution of Long COVID Severity"}\NormalTok{,}
    \AttributeTok{x =} \StringTok{"Impact on Daily Activities"}\NormalTok{,}
    \AttributeTok{y =} \StringTok{"Number of Respondents"}
\NormalTok{  ) }\SpecialCharTok{+}
  \FunctionTok{theme\_minimal}\NormalTok{()}
\end{Highlighting}
\end{Shaded}

\includegraphics{FINAL_report_Cedric_Leonard_Luigi_files/figure-latex/plot-severity-bar-1.pdf}

\begin{center}\rule{0.5\linewidth}{0.5pt}\end{center}

\subsubsection{Severity by Income Group}\label{severity-by-income-group}

Let's now explore whether income levels are associated with how severely
Long COVID impacts people.

\begin{Shaded}
\begin{Highlighting}[]
\FunctionTok{ggplot}\NormalTok{(severity\_data, }\FunctionTok{aes}\NormalTok{(}\AttributeTok{x =}\NormalTok{ SEVERITY, }\AttributeTok{fill =}\NormalTok{ RATCAT\_A)) }\SpecialCharTok{+}
  \FunctionTok{geom\_bar}\NormalTok{(}\AttributeTok{position =} \StringTok{"fill"}\NormalTok{) }\SpecialCharTok{+}
  \FunctionTok{labs}\NormalTok{(}
    \AttributeTok{title =} \StringTok{"Severity of Long COVID by Income Category"}\NormalTok{,}
    \AttributeTok{x =} \StringTok{"Severity"}\NormalTok{,}
    \AttributeTok{y =} \StringTok{"Proportion"}\NormalTok{,}
    \AttributeTok{fill =} \StringTok{"Income Category"}
\NormalTok{  ) }\SpecialCharTok{+}
  \FunctionTok{theme\_minimal}\NormalTok{() }\SpecialCharTok{+}
  \FunctionTok{scale\_fill\_brewer}\NormalTok{(}\AttributeTok{palette =} \StringTok{"Blues"}\NormalTok{)}
\end{Highlighting}
\end{Shaded}

\includegraphics{FINAL_report_Cedric_Leonard_Luigi_files/figure-latex/severity-income-plot-1.pdf}

\begin{center}\rule{0.5\linewidth}{0.5pt}\end{center}

\subsubsection{Severity by Healthcare
Barriers}\label{severity-by-healthcare-barriers}

\begin{Shaded}
\begin{Highlighting}[]
\FunctionTok{ggplot}\NormalTok{(severity\_data, }\FunctionTok{aes}\NormalTok{(}\AttributeTok{x =}\NormalTok{ SEVERITY, }\AttributeTok{fill =}\NormalTok{ MEDDL12M\_A)) }\SpecialCharTok{+}
  \FunctionTok{geom\_bar}\NormalTok{(}\AttributeTok{position =} \StringTok{"fill"}\NormalTok{) }\SpecialCharTok{+}
  \FunctionTok{labs}\NormalTok{(}
    \AttributeTok{title =} \StringTok{"Long COVID Severity by Delayed Care Due to Cost"}\NormalTok{,}
    \AttributeTok{x =} \StringTok{"Severity"}\NormalTok{,}
    \AttributeTok{y =} \StringTok{"Proportion"}\NormalTok{,}
    \AttributeTok{fill =} \StringTok{"Delayed Care (1 = Yes, 2 = No)"}
\NormalTok{  ) }\SpecialCharTok{+}
  \FunctionTok{theme\_minimal}\NormalTok{() }\SpecialCharTok{+}
  \FunctionTok{scale\_fill\_brewer}\NormalTok{(}\AttributeTok{palette =} \StringTok{"Oranges"}\NormalTok{)}
\end{Highlighting}
\end{Shaded}

\includegraphics{FINAL_report_Cedric_Leonard_Luigi_files/figure-latex/severity-healthcare-1.pdf}

\begin{center}\rule{0.5\linewidth}{0.5pt}\end{center}

\subsubsection{Why Not Use Ordinary
Regression?}\label{why-not-use-ordinary-regression}

Since \texttt{SEVERITY} has \textbf{3 ordered levels}, a standard linear
regression wouldn't be valid (because it assumes numeric distances
between categories).

Instead, we will consider:

\begin{itemize}
\tightlist
\item
  \textbf{Ordinal logistic regression} using \texttt{polr()}
  (Proportional Odds Model)
\item
  Or, if needed, \textbf{binary logistic regression} (e.g., Severe
  vs.~Not Severe)
\end{itemize}

\begin{center}\rule{0.5\linewidth}{0.5pt}\end{center}

\subsubsection{Optional: Ordinal Logistic
Regression}\label{optional-ordinal-logistic-regression}

To run this, we need the \texttt{MASS} package:

\begin{Shaded}
\begin{Highlighting}[]
\FunctionTok{library}\NormalTok{(MASS)}

\CommentTok{\# Run ordinal logistic regression}
\NormalTok{model\_severity }\OtherTok{\textless{}{-}} \FunctionTok{polr}\NormalTok{(SEVERITY }\SpecialCharTok{\textasciitilde{}}\NormalTok{ POVRATTC\_A }\SpecialCharTok{+}\NormalTok{ EDUCP\_A }\SpecialCharTok{+}\NormalTok{ USUALPL\_A }\SpecialCharTok{+} 
\NormalTok{                         MEDDL12M\_A }\SpecialCharTok{+}\NormalTok{ MEDNG12M\_A }\SpecialCharTok{+}\NormalTok{ TRANSPOR\_A }\SpecialCharTok{+}\NormalTok{ AGEP\_A,}
                       \AttributeTok{data =}\NormalTok{ severity\_data,}
                       \AttributeTok{Hess =} \ConstantTok{TRUE}\NormalTok{)}

\FunctionTok{summary}\NormalTok{(model\_severity)}
\end{Highlighting}
\end{Shaded}

\begin{verbatim}
## Call:
## polr(formula = SEVERITY ~ POVRATTC_A + EDUCP_A + USUALPL_A + 
##     MEDDL12M_A + MEDNG12M_A + TRANSPOR_A + AGEP_A, data = severity_data, 
##     Hess = TRUE)
## 
## Coefficients:
##                Value Std. Error  t value
## POVRATTC_A  -0.09841   0.029086 -3.38334
## EDUCP_A2    -0.16474   0.652830 -0.25234
## EDUCP_A3     0.43702   0.412217  1.06016
## EDUCP_A4    -0.31400   0.324009 -0.96910
## EDUCP_A5     0.03382   0.324546  0.10421
## EDUCP_A6    -0.28622   0.416922 -0.68651
## EDUCP_A7     0.08353   0.361443  0.23111
## EDUCP_A8    -0.26241   0.324065 -0.80975
## EDUCP_A9     0.21838   0.377135  0.57906
## EDUCP_A10   -0.17738   0.499826 -0.35489
## USUALPL_A2  -0.19126   0.247546 -0.77264
## USUALPL_A3  -0.01249   0.509495 -0.02451
## MEDDL12M_A2 -0.08892   0.321242 -0.27679
## MEDNG12M_A2 -0.47403   0.334079 -1.41891
## TRANSPOR_A2 -0.69649   0.206549 -3.37201
## AGEP_A       0.02301   0.005717  4.02538
## 
## Intercepts:
##                Value   Std. Error t value
## Severe|Mild    -1.0731  0.4340    -2.4725
## Mild|Some Days  1.0586  0.4339     2.4399
## 
## Residual Deviance: 1500.792 
## AIC: 1536.792
\end{verbatim}

\begin{center}\rule{0.5\linewidth}{0.5pt}\end{center}

\subsubsection{Interpreting Ordinal
Regression}\label{interpreting-ordinal-regression}

\begin{Shaded}
\begin{Highlighting}[]
\CommentTok{\# Get odds ratios and confidence intervals}
\NormalTok{ctable }\OtherTok{\textless{}{-}} \FunctionTok{coef}\NormalTok{(}\FunctionTok{summary}\NormalTok{(model\_severity))}
\NormalTok{pval }\OtherTok{\textless{}{-}} \FunctionTok{pnorm}\NormalTok{(}\FunctionTok{abs}\NormalTok{(ctable[, }\StringTok{"t value"}\NormalTok{]), }\AttributeTok{lower.tail =} \ConstantTok{FALSE}\NormalTok{) }\SpecialCharTok{*} \DecValTok{2}
\NormalTok{ci }\OtherTok{\textless{}{-}} \FunctionTok{confint}\NormalTok{(model\_severity)}
\NormalTok{ORs }\OtherTok{\textless{}{-}} \FunctionTok{exp}\NormalTok{(}\FunctionTok{cbind}\NormalTok{(}\AttributeTok{OR =} \FunctionTok{coef}\NormalTok{(model\_severity), ci))}
\NormalTok{ORs}
\end{Highlighting}
\end{Shaded}

\begin{verbatim}
##                    OR     2.5 %    97.5 %
## POVRATTC_A  0.9062794 0.8557507 0.9592034
## EDUCP_A2    0.8481167 0.2325835 3.0682577
## EDUCP_A3    1.5480819 0.6905115 3.4841812
## EDUCP_A4    0.7305214 0.3865152 1.3792268
## EDUCP_A5    1.0343979 0.5471192 1.9563801
## EDUCP_A6    0.7510973 0.3307407 1.7001672
## EDUCP_A7    1.0871211 0.5348130 2.2097827
## EDUCP_A8    0.7691938 0.4070265 1.4527422
## EDUCP_A9    1.2440654 0.5935628 2.6079402
## EDUCP_A10   0.8374575 0.3116376 2.2255572
## USUALPL_A2  0.8259143 0.5066178 1.3394109
## USUALPL_A3  0.9875906 0.3570271 2.6785637
## MEDDL12M_A2 0.9149214 0.4874123 1.7240447
## MEDNG12M_A2 0.6224895 0.3222004 1.1978508
## TRANSPOR_A2 0.4983327 0.3318474 0.7463052
## AGEP_A      1.0232780 1.0119260 1.0348689
\end{verbatim}

This gives us the odds of being in a \textbf{more severe category} of
limitation, relative to predictors.

For example:

\begin{itemize}
\tightlist
\item
  An OR \textless{} 1 for income means \textbf{higher income is
  associated with less severe symptoms}.
\item
  An OR \textgreater{} 1 for delayed care means \textbf{lack of timely
  care is associated with higher severity}.
\end{itemize}

\begin{center}\rule{0.5\linewidth}{0.5pt}\end{center}

\subsubsection{Connection to Course
Themes}\label{connection-to-course-themes}

This severity model connects to:

\begin{itemize}
\tightlist
\item
  \textbf{Ordinal data modeling} --- covered briefly in class but often
  overlooked
\item
  Health inequities: it visually shows how \textbf{low SES} groups may
  experience \textbf{worse outcomes}, not just more disease
\item
  Policy relevance: Knowing who is affected worst can guide
  \textbf{targeted interventions}
\end{itemize}

\begin{center}\rule{0.5\linewidth}{0.5pt}\end{center}

\begin{center}\rule{0.5\linewidth}{0.5pt}\end{center}

\subsubsection{Interaction Effects in Severity
Analysis}\label{interaction-effects-in-severity-analysis}

Understanding how \textbf{demographics interact} with socioeconomic
variables gives us deeper insight. For example, does \textbf{low income
affect severity more in women than in men}?

Let's explore that using an ordinal regression model that includes an
\textbf{interaction term} between \texttt{POVRATTC\_A} and
\texttt{GENDER}.

\begin{center}\rule{0.5\linewidth}{0.5pt}\end{center}

\subsubsection{Ordinal Logistic Regression with Interaction: Gender ×
Poverty
Ratio}\label{ordinal-logistic-regression-with-interaction-gender-poverty-ratio}

\begin{Shaded}
\begin{Highlighting}[]
\CommentTok{\# Run ordinal regression with interaction term}
\NormalTok{model\_interact\_sev }\OtherTok{\textless{}{-}} \FunctionTok{polr}\NormalTok{(SEVERITY }\SpecialCharTok{\textasciitilde{}}\NormalTok{ POVRATTC\_A }\SpecialCharTok{*}\NormalTok{ GENDER }\SpecialCharTok{+}\NormalTok{ EDUCP\_A }\SpecialCharTok{+} 
\NormalTok{                             USUALPL\_A }\SpecialCharTok{+}\NormalTok{ MEDDL12M\_A }\SpecialCharTok{+}\NormalTok{ MEDNG12M\_A }\SpecialCharTok{+} 
\NormalTok{                             TRANSPOR\_A }\SpecialCharTok{+}\NormalTok{ AGEP\_A,}
                           \AttributeTok{data =}\NormalTok{ severity\_data,}
                           \AttributeTok{Hess =} \ConstantTok{TRUE}\NormalTok{)}

\CommentTok{\# Summary}
\FunctionTok{summary}\NormalTok{(model\_interact\_sev)}
\end{Highlighting}
\end{Shaded}

\begin{verbatim}
## Call:
## polr(formula = SEVERITY ~ POVRATTC_A * GENDER + EDUCP_A + USUALPL_A + 
##     MEDDL12M_A + MEDNG12M_A + TRANSPOR_A + AGEP_A, data = severity_data, 
##     Hess = TRUE)
## 
## Coefficients:
##                            Value Std. Error  t value
## POVRATTC_A              -0.12228   0.045383 -2.69447
## GENDERFemale            -0.27824   0.268656 -1.03567
## EDUCP_A2                -0.14131   0.652150 -0.21668
## EDUCP_A3                 0.43819   0.411502  1.06485
## EDUCP_A4                -0.29583   0.324172 -0.91257
## EDUCP_A5                 0.05576   0.325033  0.17154
## EDUCP_A6                -0.26857   0.416595 -0.64467
## EDUCP_A7                 0.10322   0.361540  0.28551
## EDUCP_A8                -0.22638   0.325281 -0.69595
## EDUCP_A9                 0.25214   0.378604  0.66597
## EDUCP_A10               -0.17720   0.499635 -0.35466
## USUALPL_A2              -0.22348   0.248796 -0.89823
## USUALPL_A3              -0.01588   0.509312 -0.03117
## MEDDL12M_A2             -0.06630   0.322201 -0.20577
## MEDNG12M_A2             -0.50507   0.336042 -1.50298
## TRANSPOR_A2             -0.72820   0.208446 -3.49347
## AGEP_A                   0.02248   0.005735  3.91927
## POVRATTC_A:GENDERFemale  0.02987   0.054455  0.54861
## 
## Intercepts:
##                Value   Std. Error t value
## Severe|Mild    -1.3233  0.4899    -2.7013
## Mild|Some Days  0.8118  0.4880     1.6636
## 
## Residual Deviance: 1499.459 
## AIC: 1539.459
\end{verbatim}

\begin{center}\rule{0.5\linewidth}{0.5pt}\end{center}

\subsubsection{Interpreting Interaction
Terms}\label{interpreting-interaction-terms}

\begin{Shaded}
\begin{Highlighting}[]
\CommentTok{\# Odds Ratios and Confidence Intervals}
\NormalTok{ctable }\OtherTok{\textless{}{-}} \FunctionTok{coef}\NormalTok{(}\FunctionTok{summary}\NormalTok{(model\_interact\_sev))}
\NormalTok{pval }\OtherTok{\textless{}{-}} \FunctionTok{pnorm}\NormalTok{(}\FunctionTok{abs}\NormalTok{(ctable[, }\StringTok{"t value"}\NormalTok{]), }\AttributeTok{lower.tail =} \ConstantTok{FALSE}\NormalTok{) }\SpecialCharTok{*} \DecValTok{2}
\NormalTok{ci }\OtherTok{\textless{}{-}} \FunctionTok{confint}\NormalTok{(model\_interact\_sev)}
\NormalTok{ORs\_interact }\OtherTok{\textless{}{-}} \FunctionTok{exp}\NormalTok{(}\FunctionTok{cbind}\NormalTok{(}\AttributeTok{OR =} \FunctionTok{coef}\NormalTok{(model\_interact\_sev), ci))}
\NormalTok{ORs\_interact}
\end{Highlighting}
\end{Shaded}

\begin{verbatim}
##                                OR     2.5 %    97.5 %
## POVRATTC_A              0.8848988 0.8088435 0.9666915
## GENDERFemale            0.7571159 0.4467236 1.2816414
## EDUCP_A2                0.8682213 0.2382061 3.1355583
## EDUCP_A3                1.5498956 0.6923313 3.4837465
## EDUCP_A4                0.7439145 0.3934771 1.4049777
## EDUCP_A5                1.0573392 0.5587313 2.0017220
## EDUCP_A6                0.7644757 0.3368508 1.7294022
## EDUCP_A7                1.1087398 0.5453452 2.2541764
## EDUCP_A8                0.7974155 0.4209331 1.5095989
## EDUCP_A9                1.2867763 0.6121883 2.7052673
## EDUCP_A10               0.8376124 0.3118208 2.2251718
## USUALPL_A2              0.7997348 0.4894465 1.3003311
## USUALPL_A3              0.9842487 0.3559501 2.6682500
## MEDDL12M_A2             0.9358516 0.4977030 1.7670191
## MEDNG12M_A2             0.6034657 0.3110937 1.1654127
## TRANSPOR_A2             0.4827778 0.3202754 0.7256718
## AGEP_A                  1.0227332 1.0113489 1.0343555
## POVRATTC_A:GENDERFemale 1.0303253 0.9262043 1.1469234
\end{verbatim}

\begin{center}\rule{0.5\linewidth}{0.5pt}\end{center}

The interaction term \texttt{POVRATTC\_A:GENDERFemale} shows \textbf{how
poverty ratio affects women differently than men}. If the coefficient is
\textbf{significant and negative}, this suggests \textbf{female
respondents are even more sensitive to changes in poverty ratio} when it
comes to severity.

\begin{quote}
\emph{This kind of modeling is important when we suspect that health
effects are not equally distributed across social groups --- something
we've often discussed in our Biostatistics class.}
\end{quote}

\begin{center}\rule{0.5\linewidth}{0.5pt}\end{center}

\subsubsection{Visualizing Interaction (Optional but
Insightful)}\label{visualizing-interaction-optional-but-insightful}

Let's plot \textbf{predicted severity probabilities} by gender and
poverty.

\begin{Shaded}
\begin{Highlighting}[]
\FunctionTok{library}\NormalTok{(effects)}

\CommentTok{\# Compute effect}
\NormalTok{eff }\OtherTok{\textless{}{-}} \FunctionTok{Effect}\NormalTok{(}\FunctionTok{c}\NormalTok{(}\StringTok{"POVRATTC\_A"}\NormalTok{, }\StringTok{"GENDER"}\NormalTok{), model\_interact\_sev)}

\CommentTok{\# Plot it}
\FunctionTok{plot}\NormalTok{(eff,}
     \AttributeTok{main =} \StringTok{"Predicted Probability of Long COVID Severity}\SpecialCharTok{\textbackslash{}n}\StringTok{by Gender and Poverty Ratio"}\NormalTok{,}
     \AttributeTok{xlab =} \StringTok{"Poverty Ratio"}\NormalTok{,}
     \AttributeTok{ylab =} \StringTok{"Predicted Probability"}\NormalTok{,}
     \AttributeTok{multiline =} \ConstantTok{TRUE}\NormalTok{,}
     \AttributeTok{colors =} \FunctionTok{c}\NormalTok{(}\StringTok{"\#1F77B4"}\NormalTok{, }\StringTok{"\#FF7F0E"}\NormalTok{))}
\end{Highlighting}
\end{Shaded}

\includegraphics{FINAL_report_Cedric_Leonard_Luigi_files/figure-latex/severity-prediction-plot-1.pdf}

\begin{center}\rule{0.5\linewidth}{0.5pt}\end{center}

\subsubsection{Summary of Findings
(Narrative)}\label{summary-of-findings-narrative}

From this model, we see that:

\begin{itemize}
\tightlist
\item
  \textbf{Poverty ratio} continues to show a significant effect on
  severity --- lower income predicts more severe outcomes.
\item
  \textbf{Gender} is independently significant in many models.
\item
  The \textbf{interaction term} suggests that \textbf{the negative
  impact of poverty may be stronger in women}, although we'd need
  confidence intervals to confirm.
\item
  Other access-related variables (like delays due to cost or transport
  issues) remain strong predictors of increased severity.
\end{itemize}

This kind of analysis connects well with \textbf{lecture topics on
interaction terms}, \textbf{model interpretation}, and the
\textbf{real-world relevance} of statistical analysis in public health.

\begin{center}\rule{0.5\linewidth}{0.5pt}\end{center}

\end{document}
